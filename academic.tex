\documentclass[11pt,a4paper,roman]{moderncv}
\moderncvstyle{casual}
\moderncvcolor{grey}
\nopagenumbers{}

\usepackage[margin=1.25in]{geometry}

% personal data
\name{Callum}{Gray}
\title{Curriculum Vit\ae{}}
\address{116 Rutland Gardens}{London N4 1JR}{United Kingdom}
\phone[mobile]{+44 (0) 7980410478}
\email{callum.gray.10@ucl.ac.uk}
% \homepage{www.ucl.ac.uk/\textasciitilde zcapg55}
\social[github]{cuamll}
% \extrainfo{}
% \photo[64pt][0pt]{}

\begin{document}
\makecvtitle
\section{Education}
\subsection{Postgraduate}
\cventry{2014--2018}{PhD Physics}{University College London}{}{}{}
\subsection{PhD thesis}
\cvitem{title}{\emph{Field correlations in Coulomb gases}}
\cvitem{supervisors}{Prof.\ Steven Bramwell, Prof.\ Peter Holdsworth}
\cvitem{description}{I studied the phase diagram of the 2D canonical and grand canonical Coulomb gases, along with the emergent Coulomb representation of the harmonic XY model on the square lattice, and the 3D grand canonical Coulomb gas on the simple cubic lattice. I provided a Helmholtz-decomposed electric-field description of Coulombic systems and characterisations of the relevant phase transitions, along with an investigation into ``pinch-point'' features common in the correlations of dipolar systems.}

\subsection{Undergraduate}
\cventry{2010--2014}{MSci Theoretical Physics}{University College London}{\textit{First Class}}{}{}

% \subsection{Teaching}
% \cventry{2016--2018}{Postgraduate Teaching Assistant}{University College London}{}{Assisting with problem-solving tutorials and marking homework scripts for various courses. Demonstrating as part of Mathematica courses. Use of departmental online system for issuing marks and feedback to students.}{}
% \cventry{2016--2017}{Private tutoring}{Freelance}{Private tutoring of students ages 7--18 in maths and physics, including .}{}{}

\subsection{Publications}
\cventry{2018}{Field correlations of the square lattice Coulomb gas}{C. Gray, S. T. Bramwell, P. C. W. Holdsworth}{In preparation}{}{}
\cventry{2018}{A generalised theory of pinch point scattering}{C. Gray, S. T. Bramwell, P. C. W. Holdsworth}{In preparation}{}{}
\cventry{2016}{Time evolution and deterministic optimization of correlator product states}{V. Stojevic, P. Crowley, T. \DJ{}uri\'{c}, C. Gray, A. G. Green}{\url{http://dx.doi.org/10.1103/PhysRevB.94.165135}}{}{}

\section{Languages}
\cvitemwithcomment{English}{Native}{}
\cvitemwithcomment{French}{Moderate}{}

% \bigskip
% \bigskip
% \bigskip
% \bigskip
% \bigskip
% \bigskip
% \bigskip
% \bigskip
% \bigskip
% \bigskip

\section{Computer skills}
\cvitem{Fortran and C\texttt{++}}{Over the course of my PhD I have written my own programs in Fortran 2003 using standard libraries such as FFTW, LAPACK, OpenMP and OpenMPI in order to run large-scale parallelised simulations on HPC clusters. I have also upgraded and rewritten existing C\texttt{++} programs using external libraries such as Boost. I am also familiar with related utilities: \texttt{gcc}, GNU \texttt{make}, SGE for scheduling HPC jobs, etc.}
\cvitem{Python and Perl}{Experience in writing Python and Perl scripts for use with my PhD codes, for small-scale calculations, generation of input scripts, processing of output files, creation of plots using \texttt{matplotlib} as well as interfacing with \texttt{gnuplot}.}
\cvitem{Mathematica}{Significant experience; I have used Mathematica regularly for various calculations and during summer studentships, as well as demonstrating for an undergraduate course in Mathematica.}
\cvitem{Other programming}{I am familiar with MATLAB and have a limited knowledge of HTML, CSS, Javascript.}
\cvitem{General}{Good knowledge of standard utilities e.g. Linux, bash, \LaTeX, Microsoft Office, vim.}

\section{Interests}
\cvlistitem{Rock/indoor climbing}
\cvlistitem{Reading}

\section{References}
\begin{cvcolumns}
    \cvcolumn{Prof.\ Steve Bramwell}{London Centre for \\ Nanotechnology \\ s.t.bramwell@ucl.ac.uk}
    \cvcolumn{Prof.\ Peter Holdsworth}{ENS Lyon \\ peter.holdsworth@ens-lyon.fr}
    %\cvcolumn{Prof.\ Andrew Green}{London Centre for \\ Nanotechnology \\ a.green@ucl.ac.uk}
    %\cvcolumn{Prof.\ Anthony Harker}{Emeritus Professor of Physics \\ University College London \\ a.harker@ucl.ac.uk}
\end{cvcolumns}

\clearpage

% LETTER---------------------------------------------------------------------

% \recipient{recipient}{name \\ street \\ city, postcode}
%\date{\today}
%\opening{Hi Neal,}
%
%\closing{Thanks}
%\enclosure[Attached]{curriculum vit\ae{}}
%\makelettertitle
%
%% CHANGE THIS AS NECESSARY
%
%Extra information: my student number is 1025667. Don't have my NI to hand, sorry.
%
%Preferences for posts would be Python, Mathematica, PSTs, marking.
%
%\makeletterclosing

\end{document}
